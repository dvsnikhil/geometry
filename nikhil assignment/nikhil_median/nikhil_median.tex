

%% Run LaTeX on this file several times to get Table of Contents,
%% cross-references, and citations.
\documentclass[11pt]{book}
\usepackage{gvv-book}
\usepackage{gvv}
\usepackage[sectionbib,authoryear]{natbib}% for name-date citation comment the below line
\setcounter{secnumdepth}{3}
\setcounter{tocdepth}{2}

\makeindex

\begin{document}

\frontmatter

\booktitle{GEOMETRY}

\subtitle{Through Algebra}

\AuAff{D.V.S Nikhil}

\titlepage

\tableofcontents

\setcounter{page}{0}

\mainmatter

\chapter{Triangle}

\section{Median}
\begin{enumerate}[label=\thesection.\arabic*.,ref=\thesection.\theenumi]
\numberwithin{equation}{enumi}
\item If $\vec{D}$ divides $BC$ n the ratio $k:1$
        \begin{align}
		\vec{D}=\frac{kC+B}{k+1}
	\end{align}
Find the mid-points $\vec{D}, \vec{E}, \vec{F}$ of the sides $BC, CA$ and $AB$ respectively.
         \\
         \solution
Since $\vec{D}$ is the mid point of $BC$,
          \begin{align}
		  \vec{k}=1
		  \vec{D}=\frac{\vec{C}+\vec{B}}{2}=\frac{1}{2} \myvec{7 \\ 3}
	   \end{align}	  
Similarly,
		\begin{align}
			E=\frac{A+C}{2}=\frac{1}{2} \myvec{-11 \\ 2}
                 \end{align} 
		 \begin{align}
F=\frac{A+B}{2}=\frac{1}{2} \myvec{-6 \\ 3}
                 \end{align}
\item Find the equation of $AD, BE$ and $CF$.
	\\
	\solution
The direction vector of $AD$ is
	\begin{align}
	m=\vec{D}-\vec{A}
	 =\myvec{\frac{7}{2} -(-5)-\myvec{\frac{3}{2}}-(-2)}
	=\frac{1}{2} \myvec{3 \\ 7}
	=\myvec{7}{3}
	\end{align}	
Hence, the normal equation of $AD$ is
          \begin{align}
           n^T(x-A)=0
		\implies \myvec{7 & 3}x=\myvec(7 & 3) \myvec{-5 \\-2}
		=7\times -5 + 3\times -2
	        =-29
	    \end{align}	  
For $BE$,
	 \begin{align}
		 m=\vec{E}-\vec{B}
		 =\myvec{\frac{-11}{2}-(-1)} \\ \myvec{\frac{-2}{2} - (-1)}
		 =\myvec{\frac{-9}{2} \\ \frac{4}{2}}
	n=\myvec{4 \\ 9} 
	  \end{align}
Hence, the normal equation of median $BE$ is
          \begin{align}
        n^T(x-B)=0
		\implies \myvec{4 & 9}x=\myvec{4 & 9} \myvec{-1 \\ -1}
		=4\times -1 + 9 \times -1
		=-13
	  \end{align}
 For median $CF$,
	   \begin{align}
		   m=\vec{F}-\vec{C}
		   =\myvec{-\frac{-6}{2}-(-6)} \\ \myvec{\frac{-3}{2}-(4)}
		   =\myvec{\frac{6}{2} \\ \frac{-11}{2}}
	 =\myvec{6 \\ -11}
        n=\myvec{11 \\ 6}
           \end{align}		   
Hence, the normal equation of median $\myvec{CF}$ is
          \begin{align}
	   n^T(x-C)=0
		  \implies \myvec{11 & 6}x=\myvec{11& 6} \myvec{-6 \\ 4}
		  =11\times -6 + -6\times 4
		  =-42
           \end{align}
\item Find the intersection $\myvec{B}$ of $BE$ and $CF$.
	      \\
	        \solution
From $BE$ and $CF$,the above equations are:
\begin{align}
	\myvec{4 9}x=-13
	\myvec{11 6}x=-42
\end{align}
From the above, the augmented matrix is
\begin{align}
\myvec{
4 & 9 & -13 \\
11 & 6 & -42
}
	\xleftrightarrow[]{R_1} \leftarrow {\frac{R_1}{4}}	
\end{align}		  
The matrix can be reduced as follows:
\begin{align}
   \label{eq:matthrowoperations}
\myvec{
1 & \frac{9}{4} & \frac{-13}{4} \\
11 & 6 & -42 \\
}
	\xleftrightarrow[]{R_2} \leftarrow {R_2-11R_1}
\myvec{
1 & \frac{9}{4} & \frac{13}{4} \\
0 & \frac{-75}{4} & \frac{-25}{4} \\
}
	\xleftrightarrow[]{R_2} \leftarrow {\frac{-4R_2}{75}}
\myvec{
1 & \frac{9}{4} & \frac{-13}{4} \\
0 & 1 & \frac{1}{3} \\
}
	\xleftrightarrow[]{R_2} \leftarrow {R_1-\frac{9R_2}{4}}
\myvec{
1 & 0 & -4 \\
0 & -1 & \frac{1}{3} \\
	}
Using Gauss elimination,
\vec{G}=\myvec{-4 \\ \frac{1}{3}}
\end{align}
\item Verify that
	\begin{align}
\frac{BG}{GE}=\frac{CG}{GF}=\frac{AG}{GD}=2
         \end{align}
            \\ 
             \solution
In order to find $BG$, we need to find
$\vec{G}-\vec{B}$ and in order to find $GE$, we need to find $\vec{E}-\vec{G}$.
\begin{align}
	BG&=\vec{G}-\vec{B}=\myvec{-4 -(-1)} \\ \myvec{\frac{1}{3}-(-1)} \\
=\myvec{-3 \\ \frac{4}{3}} \\
	GE&=\vec{E}-\vec{G}=\myvec{\frac{-11}{2}-(-4)} \\ \myvec{\frac{2}{2} - \frac{1}{3}}=\myvec{\frac{-3}{2} \\ \frac{4}{6}} \\
Therefore, \frac{BG}{GE}=2 \\ 
\end{align}
In order to find $GF$, we need to find $\vec{F}-\vec{G}$ and in order to find $CG$
, we need to find $\vec{G}-\vec{C}$.
\begin{align}
	GF&=\vec{F}-\vec{G}=\myvec{\frac{-6}{2}-(-4)} \\ \myvec{\frac{-3}{2} - \frac{1}{3}}= \myvec{\frac{2}{2} \\ \frac{-11}{6}} \\
	CG&=\vec{G}-\vec{C}=\myvec{-4 - (-6)} \\ \myvec{\frac{1}{3}-(4)}=\myvec{2 \\ \frac{-11}{3}} \\
Therefore, \frac{CG}{GF}=2
\end{align}
In order to find $AG$, we need to find $
\vec{G}-\vec{A}$ and in order to find $GD$
, we need to find $\vec{D}-\vec{G}$.
\begin{align}
	AG&=\vec{G}-\vec{A}=\myvec{-4 -(-5)} \\ \myvec{\frac{1}{3}-(-2)}=\myvec{1 \\ \frac{7}{3}} \\
	GD&=\vec{D}-\vec{G}=\myvec{\frac{-7}{2}-(-4)} \\ \myvec{\frac{3}{2}-\frac{1}{3}}=\myvec{\frac{1}{2} \\ \frac{7}{6}} \\
Therefore, \frac{AG}{GD}=2 \\
\implies \frac{BG}{GE}=\frac{CG}{GF}=\frac{AG}{GD}=2 \\
Hence verified \\
\end{align}
\item Show that $\vec{A}, \vec{G}$ and $\vec{D}$ are collinear if
	\\ 
	 \solution
Points $\vec{A}, \vec{D}, \vec{G}$ are defined to be collinear if
	\begin{align}
        \text{rank}\myvec{
		1 & 1 & 1 \\
	       \vec{A} & \vec{D} & \vec{G}\\
                }&=2
        \end{align}	
\begin{align}
\myvec{
1 & 1 & 1 \\
-5 & \frac{-7}{2} & -4 \\
-2 & \frac{3}{2} & \frac{1}{3} \\
}
	\xleftrightarrow[]{R_2} \leftarrow {R_3+5R_1}
\end{align}
\label{eq:matthrowoperations}
The matrix can be reduced as follows:
\begin{align}
\myvec{
1 & 1 & 1 \\
0 & \frac{3}{2} & 1 \\
-2 & \frac{3}{2} & \frac{1}{3} \\
}
	\xleftrightarrow[]{R_3} \leftarrow {R_3+2R_1}
\myvec{
1 & 1 & 1 \\
0 & \frac{3}{2} & 1 \\
0 & \frac{7}{2} & \frac{7}{3} \\
}
	\xleftrightarrow[]{R_3} \leftarrow{R_3-\frac{7R_2}{3}}
\myvec{
1 & 1 & 1 \\
0 & \frac{3}{2} & 1 \\
0 & 0 & 0 \\
}
\end{align}
\item Verify that
	\begin{align}
		\vec{G}=\frac{\vec{A+B+C}}{3}	
	\end{align}	
	\begin{figure}
		\includegraphics[width=\columnwidth]{/sdcard/D.V.S.NIKHIL/figs/D.V.S.NIKHIL_figures/nikhil_image}
	\caption{Triangle}
	\label{fig1.2:Triangle}
	\end{figure}
	     \\
              \solution  
	      \begin{align}
		      \vec{G}&=\frac{\vec{A+B+C}}{3}
		      \\
		      \vec{G}&=\frac{1}{3}\brak{\myvec{-5\\-2}+\myvec{-1\\-1}+\myvec{-6\\4}}
		      \\
		      &=\frac{1}{3}\myvec{-12\\1}
		      \\
		      &=\myvec{-4 \\ \frac{1}{3}}	     
		      \\
		      \implies \vec{G}&=\frac{\vec{A+B+C}}{3}	      
               \end{align}     
	       Hence verifed\\
\item Verify that
	\begin{align}
		\vec{A}-\vec{F}=\vec{E}-\vec{D}
	\end{align}
	%\iffalse
	\begin{figure}
		\includegraphics[width=\columnwidth]{/sdcard/D.V.S.NIKHIL/figs/D.V.S.NIKHIL_figures/nikhil_image2}
		\caption{Triangle}
		\label{fig1.3:Triangle}
	\end{figure}
	%\fi
         \\
	 \solution
\begin{align}
	\vec{A}-\vec{F}=\myvec{(-5) - \frac{-6}{2}} \\ 
	\myvec{(-2) -\frac{-3}{2}}=\myvec{(-2) \\ \frac{-1}{2}}
	\vec{E}-\vec{D}=\myvec{\frac{11}{2}-\frac{-7}{2}}\\         	\myvec{\frac{2}{2}-\frac{3}{2}}
	=\myvec{{-2} \\ \frac{-1}{2}}
	\implies \vec{A}-\vec{F}=\vec{E}-\vec{D}	
\end{align}
Hence verified\\
\end{enumerate}
\end{document}









   
	    

		 
	    

