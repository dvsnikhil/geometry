%%Run LaTeX on this file several times to get Table of Contents,
%% cross-references, and citations.
\documentclass[11pt]{book}
\usepackage{gvv-book}
\usepackage{gvv}
\usepackage[sectionbib,authoryear]{natbib}% for name-date citation comment the below line
\setcounter{secnumdepth}{3}
\setcounter{tocdepth}{2}
\makeindex
\begin{document}
\frontmatter
\booktitle{GEOMETRY}
\subtitle{Through Algebra}
\AuAff{D.V.S. NIKHIL}
\titlepage
\tableofcontents
\setcounter{page}{1}
\mainmatter
\chapter{Triangle}
Consider a triangle with vertices
\begin{align}
\vec{A} = \myvec{-5 \\ -2},\,
\vec{B} = \myvec{-1 \\ -1},\,
\vec{C} = \myvec{-6 \\ 4}
\end{align}
\section{Vectors}
\begin{enumerate}[label=\thesection.\arabic*.,ref=\thesection.\theenumi]
\numberwithin{equation}{enumi}
\item the direction vector of $AB$ is defined as 
\begin{align}
 \vec{B}-
  \vec{A}
\end{align}
Find the direction vectors of $AB, BC$ and $CA$.
\\
\solution
\begin{enumerate} 
\item  The Direction vector of $AB$ is 
\begin{align}&= \vec{B} - \vec{A} \\::
 &= \myvec{ -1  -(-5)\\ -1 - (-2) } \\&= \myvec{4\\ 1 }
 \end{align}
\item The Direction vector of $BC$ 
\begin{align}&= \vec{C} - \vec{B}\\
 &= \myvec{ -6 - (-1)\\ 4 - (-1) } \\&= \myvec{-5\\ 5 }
  \end{align}
  \item  The Direction vector of $CA$  
  \begin{align} &= \vec{A} - \vec{C} \\ 
 &= \myvec{ -5 - (-6)\\ -2 - (4) } \\&= \myvec{ 1 \\ -6 }
  \end{align}
 \end{enumerate}
 \item The length of side $AB,BC$ and $AC$ is
\solution
Given, 
\begin{align}
\vec{A} = \myvec{-5\\-2},\\
\vec{B} = \myvec{-1\\-1},\\
\vec{C} = \myvec{-6\\4} 
\end{align}
Now solving for $AB$,\\
\begin{align}
\norm{\vec{A}-\vec{B}}\ &=  \sqrt{\brak{\vec{A}-\vec{B}}^{\top}\brak{\vec{A}-\vec{B}}} \\
	\vec{A}-\vec{B} &= \myvec{-5\\-2} - \myvec{-1\\-1} &= \myvec{-4\\-1}\\
\norm{\vec{A}-\vec{B}} &= \sqrt{\myvec{-4 & -1}\myvec{-4\\-1}}\\
&= \sqrt{\brak{4}^2 +\brak{1}^2}\\
	\implies \norm{\vec{A}-\vec{B}} &=\sqrt{17}
\end{align}
Now solving for $BC$,\\
\begin{align}
	\norm{\vec{B}-\vec{C}}\ &=  \sqrt{\brak{\vec{B}-\vec{C}}^{\top}\brak{\vec{B}-\vec{C}}} \\
\vec{B}-\vec{C} &= \myvec{5\\-5}\\
\norm{\vec{B}-\vec{C}} &= \sqrt{\myvec{5 & -5}\myvec{5\\-5}}\\
&= \sqrt{\brak{5}^2+\brak{5}^2}\\
\implies \norm{\vec{B}-\vec{C}} &= \sqrt{50}
\end{align}
Now solving for $AC$,\\
\begin{align}
	\norm{\vec{A}-\vec{C}}\ &=  \sqrt{\brak{\vec{A}-\vec{C}}^{\top}\brak{\vec{A}-\vec{C}}} \\
\vec{A}-\vec{C} &= \myvec{1\\-6}\\
\norm{\vec{A}-\vec{C}} &= \sqrt{\myvec{1 & -6}\myvec{1\\-6}}\\
&= \sqrt{\brak{1}^2+\brak{6}^2}\\
\implies \norm{\vec{A}-\vec{C}}&=\sqrt{37}
\end{align}
\item Points $\vec{A}, \vec{B}, \vec{C}$ are defined to be colliner if
	\begin{align}
		\rank{\myvec{1 & 1 & 1 \\ \vec{A} & \vec{B} & \vec{C}}} = 2
	\end{align}
\solution\\
Given that,
\begin{align}
    \vec{A} = \myvec{-5\\-2}
    \quad
    \vec{B} &= \myvec{-1\\-1}
    \quad
    \vec{C} = \myvec{-6\\4}
\end{align}
Given that $\vec{A},\vec{B},\vec{C}$ are collinear if
\begin{align}
    \text{rank}\myvec{
    1 & 1 & 1\\
    \vec{A} & \vec{B} & \vec{C} \\
    } &< 3 
    \label{eq:1.1.3.2}
\end{align} 
Let
\begin{align}
\vec{R} &= \myvec{
1 & 1 & 1 \\
-5 & -1 & -6 \\
-2 & -1 & 4
} \\
\end{align}
The matrix $\vec{R}$ can be row reduced as follows,
\begin{align}
\label{eq:matthrowoperations}
\myvec{
1 & 1 & 1 \\
0 & 4 & -1 \\
-2 & -1 & -4
}
\xleftrightarrow[]{R_3 \leftarrow 2R_1+R_3}
\myvec{
1 & 1 & 1 \\
0 & 4 & -1 \\
0 & 1 & 6
}\\
\xleftrightarrow[]{R_3 \leftarrow R_3 - \frac{1}{4}R_2}
\myvec{
1 & 1 & 1 \\
0 & 4 & -1 \\
0 & 0 & \frac{25}{4}
}\\
\xleftrightarrow[]{R_2 \leftarrow \frac{1}{4}R_2}
\myvec{
1 & 1 & 1 \\
0 & 1 & \frac{-1}{4} \\
0 & 0 & \frac{25}{4}
}
\end{align}


There are no zero rows. So,
\begin{align}
    \text{rank}\myvec{
    1 & 1 & 1\\
    \vec{A} & \vec{B} & \vec{C} \\
    } &= 3 
\end{align}  
Hence, from \eqref{eq:1.1.3,2} the points $\vec{A},\vec{B},\vec{C}$ are not collinear. 
%\iffalse
\begin{figure}[H]
%\includegraphics[width=\columnwidth]{/sdcard/Documents/figs/Collinear.png}
\caption{$\vec{A},\vec{B},\vec{C}$ plot}
\label{fig1:Triangle}
\end{figure}
From Fig. \ref{fig1:Triangle}, We can see that $\vec{A},\vec{B},\vec{C}$ are not collinear .
\item The parametric form of the equation of $AB$ is
	\begin{align}
		\vec{x}=\vec{A}+k\vec{m}
	\end{align}
	where\\
	\begin{align}
		\vec{m} = \vec{B} - \vec{A}
	\end{align}
	is the direction vector of $AB$.
	Find the parametric equations of $AB,BC$ and $CA$.
        \solution\\
	The parametric equation for $AB$ is given by
	\begin{align}
		\vec{x} &= \vec{A} + k\vec{m}\\
		\text{where, } \vec{m} &= \vec{B} -\vec{A}\\
		&= \myvec{-1 \\ -1} -\myvec{-5\\ -2}\\
		&= \myvec{4 \\ 1}
	\end{align}
	Hence we get,
	\begin{align}
		\vec{AB}: \vec{x} = &\myvec{-5\\-2} + k \myvec{4\\1}
	\end{align}
	Similarly, 
	\begin{align}
		\vec{BC}: \vec{x} = &\myvec{-1\\-1} + k \myvec{5\\-5}\\
		\vec{CA}: \vec{x} = &\myvec{-6\\4} + k \myvec{1\\-6}
	\end{align}
\item The normal form of the equation of $\vec{AB}$ is
\begin{align}
\vec{n}^{\top}\myvec{\vec{x}-\vec{A}}=0
\end{align}
where
\begin{align}
\vec{n}^{\top}\vec{m}&=\vec{n}^{\top}\myvec{\vec{B}-\vec{A}}=0
\end{align} 
or,\begin{align}
\vec{n}&=\myvec{0 &1 \\-1 & 0}\vec{m}
\end{align}
then find the normal form of hyuythe equations of $\vec{AB}$ $\vec{BC}$ and $\vec{CA}$
\solution:\\
       The normal equation for the side $AB$ is
\begin{align}
\vec{n}^{\top}\myvec{\vec{x}-\vec{A}}&=0\\
\implies
\vec{n}^{\top}\vec{x}&=\vec{n}^{\top}\vec{A}
\end{align}
Now our task is to find the $\vec{n}$ so that we can find $\vec{n}^{\top}$.
As given. 
\begin{align}
  \vec{n} &= \myvec{0 & 1\\
  -1 & 0}\vec{m}
\end{align}
Here $\vec{m} = \vec{B}- \vec{A}$ for side $\vec{AB}$
\begin{align}
\implies
\vec{m}&=\myvec{-1\\-1} - \myvec{-5\\-2}\\
&=\myvec{4\\1}
\end{align}
Now as we have obtained vector $\vec{m}$.we can use this to obtain vector $\vec{n}$
\begin{align}
\vec{n} &= \myvec{0 & 1\\
  -1 & 0}\myvec{4\\1}
 = \myvec{1\\4}
\end{align}
The transpose of $\vec{n}$ is
\begin{align}
  \vec{n}^{\top}&=\myvec{1 & 4}
\end{align}
Hence the normal equation of side $AB$ is 
\begin{align}
    \myvec{1 & 4}\vec{x}&=\myvec{1 & 4}\myvec{-5\\-2}\\
    \implies \myvec{1 & -4}\vec{x} &= -13
\end{align}
%\iffalse
\begin{figure}[H]
%\includegraphics [width=\columnwidth] {/sdcard/Documents/figs/AB.png}
\caption{ The line $\vec{AB}$ plotted}
\label{fig:line AB}
\end{figure}
%\fi
Similarly
\begin{align}
	\implies
	\vec{BC:} \myvec{5 & -5}\vec{x} &=-50
\end{align}
%\iffalse
\begin{figure}[H]
%\includegraphics [width=\columnwidth] {/sdcard/Documents/figs/BC.png}
\caption{ The line $\vec{BC}$ plotted}
\label{fig:line BC}
\end{figure}
%\fi
\begin{align}
	\implies \vec{CA:} \myvec{-6 & 1}\vec{x} &= -32
\end{align}
%\iffalse
\begin{figure}[H]
%\includegraphics [width=\columnwidth] {/sdcard/Documents/figs/CA.png}
\caption{ The line $\vec{CA}$ plotted}
\label{fig:line CA}
\end{figure}
%\fi
\item Find the area of the $\triangle ABC$
	\solution\\
Given
\begin{align}
 \vec{A} = \myvec{-5\\-2};
 \vec{B} = \myvec{-1\\-1};
 \vec{C} = \myvec{-6\\4}
 \end{align}
 \begin{align}
 \vec{A}-\vec{B} &= \myvec{-5\\-2} - \myvec{-1\\-1} = \myvec{-4\\-1}\\
 \vec{A}-\vec{C} &= \myvec{-5\\-2} - \myvec{-6\\4} = \myvec{1\\-6}\\
\therefore(\vec{A}-\vec{B})\times(\vec{A}-\vec{C}) 
 &= \mydet{-4 & -1\\1 & -6}\\
 &= -4 \times -6 - 1 \times (1)\\ &= 24 - 1\\ &=23 \\
 \implies\frac{1}{2}\norm{(\vec{A}-\vec{B})\times(\vec{A}-\vec{C})} &= \frac{1}{2}\norm{23}= \frac{23}{2}
\end{align}
\item Find the angles $A, B, C$ if 
    \label{prop:angle2d}
  \begin{align}
    \label{eq:angle2d}
   \cos A \triangleq 
\frac{\brak{\vec{B}-\vec{A}}^{\top}{\vec{C}-\vec{A}}}{\norm{\vec{B}-\vec{A}}\norm{\vec{C}-\vec{A}}}
  \end{align}
\solution\\
From the given values of $\vec{A},\vec{B},\vec{C}$,\\
\begin{enumerate}
 \item Finding the value of angle A
\begin{align}
 \vec{B}-\vec{A} &=\myvec{4\\1}
\end{align}
and 
\begin{align}
 \vec{C}-\vec{A} &= \myvec{-1\\6}
\end{align}
also calculating the values of norms
\begin{align}
 \norm{\vec{B}-\vec{A}} &= \sqrt{17}\\
 \norm{\vec{C}-\vec{A}} &= \sqrt{37}
\end{align}
and by doing matrix multiplication we get,
\begin{align}
\begin{split}
 (\vec{B}-\vec{A})^{\top}(\vec{C}-\vec{A}) &= \myvec{4&1}\myvec{-1\\6} =2 
\end{split}
\end{align}
So, we get
\begin{align}
 \cos{A} &= \frac{2}{\sqrt{17} \sqrt{37}}\\
 &= \frac{2}{\sqrt{629}}\\
 \implies A& = \cos^{-1}{\frac{2}{\sqrt{629}}}
\end{align}
\item Finding the value of angle B
\begin{align}
 \vec{C}-\vec{B} &=\myvec{-5\\5}
\end{align}
and 
\begin{align}
 \vec{A}-\vec{B} &= \myvec{-4\\-1}
\end{align}
also calculating the values of norms
\begin{align}
 \norm{\vec{C}-\vec{B}} &= \sqrt{50}
 \norm{\vec{A}-\vec{B}} &= \sqrt{17}
\end{align}
and by doing matrix multiplication we get,
\begin{align}
\begin{split}
 (\vec{C}-\vec{B})^{\top}(\vec{A}-\vec{B}) &= \myvec{-5&5}\myvec{-4\\-1} = -25
\end{split}
\end{align}
So, we get 
\begin{align}
	\cos{B} &= \frac{-25}{{50} \sqrt{17}}\\
 &= \frac{-25}{\sqrt{850}}\\
 \implies B& = \cos^{-1}{\frac{-25}{\sqrt{850}}}
\end{align}
\item Finding the value of angle C
\begin{align}
 \vec{A}-\vec{C} &=\myvec{1\\-6}
\end{align}
and 
\begin{align}
 \vec{B}-\vec{C} &= \myvec{5\\-5}
\end{align}
also calculating the values of norms
\begin{align}
 \norm{\vec{A}-\vec{C}} &= \sqrt{37}\\
	\norm{\vec{B}-\vec{C}} &= \sqrt{50}
\end{align}
and by doing matrix multiplication we get,
\begin{align}
\begin{split}
 (\vec{A}-\vec{C})^{\top}(\vec{B}-\vec{C}) &= \myvec{1&-6}\myvec{5\\-5}\\
 &= 35
\end{split}
\end{align}
so 
\begin{align}
	\cos{C} &= \frac{35}{{37} \sqrt{50}}\\
 &= \frac{35}{\sqrt{1850}}\\
 \implies C &= \cos^{-1}{\frac{35}{\sqrt{1850}}}
\end{align}
\end{enumerate}
\latexprintindex
\end{enumerate}
\end{document}
